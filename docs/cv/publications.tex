%-------------------------------------------------------------------------------
%	SECTION TITLE
%-------------------------------------------------------------------------------
\cvsection{Publications}


%-------------------------------------------------------------------------------
%	CONTENT
%-------------------------------------------------------------------------------

%-------------------------
\bigskip
Undergraduate Researchers*, Co-lead Authors$^\dagger$

\smallskip
\textit{in review}
\smallskip

\begin{cvitems}

\item Kaul, RajReni B.$^\dagger$, \textcolor{awesome}{Michelle V. Evans}$^\dagger$, Courtney C. Murdock, John M. Drake. Spatio-temporal spillover risk of yellow fever in Brazil.
\smallskip

\item \textcolor{awesome}{Evans, Michelle V.}, Justine C. Shiau, Nicole Solano*, Melinda A. Brindley, John M. Drake, Courtney C. Murdock. Carry-over effects of larval microclimate on the transmission potential of a mosquito-borne pathogen.

\end{cvitems}

\bigskip
\textit{published}
\smallskip

\begin{etaremune}

\item \textcolor{awesome}{Evans, Michelle V.}, Courtney C. Murdock, John M. Drake. 2018. Anticipating emerging mosquito-borne flaviviruses in the USA: What comes after Zika? \textit{Trends in Parasitology. In press.}

\item Murdock, Courtney, \textcolor{awesome}{Michelle V. Evans}, Taylor McClanahan*, Kerri Miazgowicz, and Blanka Tesla. 2017. Fine-scale variation in microclimate across an urban landscape changes the capacity of \textit{Aedes albopictus} to vector arboviruses. \textit{PLoS Neglected Tropical Diseases}, doi:10.1371/journal.pntd.0005640.
\smallskip

\item Mordecai, Erin, Jeremy Cohen, \textcolor{awesome}{Michelle V. Evans}, Prithvi Gudapati, Leah R. Johnson, Catherine A. Lippi, Kerri Miazgowicz, et al. 2016. Detecting the impact of temperature on transmission of Zika, dengue and chikungunya using mechanistic models. \textit{PLoS Neglected Tropical Diseases}, doi:10.1101/063735.
\smallskip

\item \textcolor{awesome}{Evans, Michelle V.}, Tad A. Dallas, Barbara A. Han, Courtney C. Murdock, and John M. Drake. 2017. Data-driven identification of potential Zika virus vectors. \textit{eLife} 6: e22053. doi:10.7554/eLife.22053.
\end{etaremune}

\bigskip
