%-------------------------------------------------------------------------------
%	SECTION TITLE
%-------------------------------------------------------------------------------
\cvsection{Publications}


%-------------------------------------------------------------------------------
%	CONTENT
%-------------------------------------------------------------------------------

%-------------------------
\bigskip
Undergraduate Researchers*, Co-lead Authors$^\dagger$

\smallskip

\begin{etaremune}

\item Wimberly, Michael, Justin K. Davis, \textcolor{awesome}{Michelle V. Evans}, Andrea Hess, Philip M. Newberry, Nicole Solano-Asamoah, Courtney C. Murdock. 2020. Land cover affects microclimate and temperature suitability for arbovirus transmission in an urban landscape. \textit{PLoS Neglected Tropical Diseases} 14(9):e008614. doi:10.1371/journal.pntd.0008614. \smallskip

\item \textcolor{awesome}{Evans, Michelle V.}, Andres Garchitorena, Rado J.L. Rakotonanahary, John M. Drake, Benjamin Andriamihaja, Elinambinina Rajaonarifara, Calistus N. Ngonghala, Benjamin Roche, Matthew H. Bonds, Julio Rakotonirina. 2020. Reconciling model predictions with low reported cases of COVID-19 in Sub-saharan Africa: Insights from Madagascar. \textit{Global Health Action} 13(1):1816044. doi:10.1080/16549716.2020.1816044. \smallskip

\item \textcolor{awesome}{Evans, Michelle V. }, Philip M. Newberry, Courtney C. Murdock. 2020. Carry-over effects of the larval environment in mosquito-borne disease systems. In: J. M. Drake, M. Strand, M. Bonsall (Eds.), \textit{Population Biology of Vector-Borne Diseases.} Oxford University Press. \textit{Book Chapter}.\smallskip

\item Reitmayer, Christine M., \textcolor{awesome}{Michelle V. Evans}, Kerri L. Miazgowicz, Philip M. Newberry, Nicole Solano-Asamoah, Blanka Tesla, and Courtney C. Murdock. 2020. Mosquito-virus Interactions. In: J. M. Drake, M. Strand, M. Bonsall (Eds.), \textit{Population Biology of Vector-Borne Diseases.} Oxford University Press. \textit{Book Chapter}. \smallskip

\item \textcolor{awesome}{Evans, Michelle V.}, Carl W. Hintz*, Lindsey Jones*, Justine Shiau, Nicole Solano, John M. Drake, Courtney C. Murdock. 2019. Microclimate and larval habitat density predict adult \textit{Aedes albopictus} abundance in urban areas. \textit{The American Journal of Tropical Medicine and Hygiene}. doi:10.4269/ajtmh.19-0220. \smallskip

\item Kaul, RajReni B.$^\dagger$, \textcolor{awesome}{Michelle V. Evans}$^\dagger$, Courtney C. Murdock, John M. Drake. 2018. Spatio-temporal spillover risk of yellow fever in Brazil. \textit{Parasites \& Vectors} 11:488. doi: 10.1186/s13071-018-3063-6.
\smallskip

\item \textcolor{awesome}{Evans, Michelle V.}, Justine C. Shiau, Nicole Solano*, Melinda A. Brindley, John M. Drake, Courtney C. Murdock. 2018. Carry-over effects of urban larval environments on the transmission potential of dengue-2 virus. \textit{Parasites \& Vectors} 11:426. doi:10.1186/s13071-018-3013-3.

\item \textcolor{awesome}{Evans, Michelle V.}, Courtney C. Murdock, John M. Drake. 2018. Anticipating emerging mosquito-borne flaviviruses in the USA: What comes after Zika? \textit{Trends in Parasitology} 34(7):544. \linebreak doi:10.1016/j.pt.2018.02.010

\item Murdock, Courtney C., \textcolor{awesome}{Michelle V. Evans}, Taylor McClanahan*, Kerri Miazgowicz, and Blanka Tesla. 2017. Fine-scale variation in microclimate across an urban landscape changes the capacity of \textit{Aedes albopictus} to vector arboviruses. \textit{PLoS Neglected Tropical Diseases} 11(5)e0005640:, \linebreak doi:10.1371/journal.pntd.0005640.
\smallskip

\item Mordecai, Erin, Jeremy Cohen, \textcolor{awesome}{Michelle V. Evans}, Prithvi Gudapati, Leah R. Johnson, Catherine A. Lippi, Kerri Miazgowicz, et al. 2017. Detecting the impact of temperature on transmission of Zika, dengue and chikungunya using mechanistic models. \textit{PLoS Neglected Tropical Diseases} 11(4):e0005568, \linebreak doi:10.1101/063735.
\smallskip

\item \textcolor{awesome}{Evans, Michelle V.}, Tad A. Dallas, Barbara A. Han, Courtney C. Murdock, and John M. Drake. 2017. Data-driven identification of potential Zika virus vectors. \textit{eLife} 6: e22053. doi:10.7554/eLife.22053.
\end{etaremune}

\bigskip
