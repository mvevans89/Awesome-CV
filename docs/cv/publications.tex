%-------------------------------------------------------------------------------
%	SECTION TITLE
%-------------------------------------------------------------------------------
\cvsection{Publications}


%-------------------------------------------------------------------------------
%	CONTENT
%-------------------------------------------------------------------------------

%-------------------------
\bigskip
Undergraduate Researchers*, Co-lead Authors$^\dagger$

\smallskip

\begin{etaremune}

\item Portois, Julie D., Krti Tallam, Isabel Jones, Elizabeth Hyde, Andrew Chamberlin, \textcolor{awesome}{Michelle V. Evans}, Felana A. Ihantamalala, Laura F. Cordier, B\'{e}n\'{e}dicte R. Razafinjato, Rado J.L. Rakotonanahary, Andritiana Tsirinomen'ny Aina, Patrick Soloniaina, Sahondraritera H. Raholiarimanana, Celestin Razafinjato, Matthew H. Bonds, Giulio A. De Leo, Susanne Sokolow, Andres Garchitorena. 2023. Climate, land-use and socio-economic factors can predict malaria dynamics at fine spatial scales relevant to local health actors: evidence from rural Madagascar. \textit{PLoS Global Public Health}. In press. \smallskip

\item \textcolor{awesome}{Evans, Michelle V.}$^\dagger$, Tanjona Andr\'{e}ambeloson$^\dagger$, Mauricianot Randriamihaja, Felana A. Ihatamalala, Laura Cordier, Giovanna Cowley, Karen Finnegan, Feno Hanitriniaina, Ann C. Miller, Lanto Marovavy Ralantomalala, Andry Randriamahasoa, B\'{e}n\'{e}dicte R. Razafinjato, Emeline Razanahanitriniaina, Rado J.L. Rakotonanahary. Isa\"{i}e Jules Andriamiandra, Matthew H. Bonds. Andres Garchitorena. 2022. Geographic barriers to care persist at the community healthcare level: evidence from rural Madagascar. \textit{PLoS Global Public Health} 2(12):e0001028. doi: 10.1371/journal.pgph.0001028. \smallskip

\item \textcolor{awesome}{Evans, Michelle V.} and John M. Drake. 2022. A data-driven horizon scan of bacterial pathogens at the wildlife-livestock interface. \textit{EcoHealth} 19(2):246-258. doi: 10.1007/s10393-022-01599-3. \smallskip

\item Rasambainarivo, Fidisoa, Tanjona Ramiadantsoa, Antso Raherinandrasana, Santatra Randrianarisoa, Benjamin L. Rice, \textcolor{awesome}{Michelle V. Evans}, Benjamin Roche, Fidiniaina Mamy Randriatsarafara, Amy Wesolowski, Jessica C. Metcalf. Prioritizing COVID-19 vaccination efforts and dose allocation within Madagascar. \textit{BMC Public Health}. doi:10.1186/s12889-022-13150-8. \smallskip

\item \textcolor{awesome}{Evans, Michelle V.}, Siddharth Bhatnagar, John M. Drake, Courtney C. Murdock, Shomen Mukherjee. 2022. Socio-ecological dynamics in urban systems: An integrative approach to mosquito-borne disease in Bengaluru, India. \textit{People and Nature} 4(3):730-743. doi: 10.1002/pan3.10311. \smallskip

\item Russell, Marie C.$^\dagger$, Catherine M. Herzog$^\dagger$, Zachary Gajewski, Chloe Ramsay, Fadoua El Moustaid, \textcolor{awesome}{Michelle V. Evans}, Tishna Desai, Nicole L. Gottdenker, Sara L. Hermann, Alison G. Power, Andrew C. McCall. 2022. Both consumptive and non-consumptive effects of predators impact mosquito populations and have implications for disease transmission. \textit{eLife} e71503. doi: 10.7554/eLife.71503. \smallskip

\item Rakotonanahary, Rado J.L., Herinjaka Andriambolamanana, Benedicte Razafinjato, Estelle M. Raza-Fanomezanjanahary, Vero Ramanandraitsiory, et al. Matthew H. Bonds. 2021. Integrating health systems and science to respond to COVID-19 in a model district of rural Madagascar. \textit{Frontiers in Public Health} 9:654299. doi: 10.3389/fpubh.2021.654299 \smallskip

\item \textcolor{awesome}{Evans, Michelle V.}, John M. Drake, Lindsey Jones*, Courtney C. Murdock. 2021. Assessing temperature-dependent competition between two invasive mosquito species. \textit{Ecological Applications} e02334. \\ doi: 10.1002/eap.2334. \smallskip

\item \textcolor{awesome}{Evans, Michelle V.}, Matthew H. Bonds, Laura F. Cordier, John M. Drake, Felana Ihantamalala, Justin Haruna, Ann C. Miller, Courtney C. Murdock, Marius Randriamanambtsoa, Estelle M. \\ Raza-Fanomezanjanahary, Bénédicte R. Razafinjato, Andres C. Garchitorena. 2021. Socio-demographic, not environmental, risk factors explain fine-scale spatial patterns of diarrhoeal disease in Ifanadiana, rural Madagascar. \textit{Proceedings of the Royal Society B} 288:20202501. doi:10.1098/rspb.2020.2501. \smallskip

\item Wimberly, Michael, Justin K. Davis, \textcolor{awesome}{Michelle V. Evans}, Andrea Hess, Philip M. Newberry, Nicole Solano-Asamoah, Courtney C. Murdock. 2020. Land cover affects microclimate and temperature suitability for arbovirus transmission in an urban landscape. \textit{PLoS Neglected Tropical Diseases} 14(9):e008614. doi:10.1371/journal.pntd.0008614. \smallskip

\item \textcolor{awesome}{Evans, Michelle V.}, Andres Garchitorena, Rado J.L. Rakotonanahary, John M. Drake, Benjamin Andriamihaja, Elinambinina Rajaonarifara, Calistus N. Ngonghala, Benjamin Roche, Matthew H. Bonds, Julio Rakotonirina. 2020. Reconciling model predictions with low reported cases of COVID-19 in Sub-saharan Africa: Insights from Madagascar. \textit{Global Health Action} 13(1):1816044. \\doi:10.1080/16549716.2020.1816044. \smallskip

\item \textcolor{awesome}{Evans, Michelle V. }, Philip M. Newberry, Courtney C. Murdock. 2020. Carry-over effects of the larval environment in mosquito-borne disease systems. In: J. M. Drake, M. Strand, M. Bonsall (Eds.), \textit{Population Biology of Vector-Borne Diseases.} Oxford University Press. \textit{Book Chapter}.\smallskip

\item Reitmayer, Christine M., \textcolor{awesome}{Michelle V. Evans}, Kerri L. Miazgowicz, Philip M. Newberry, Nicole Solano-Asamoah, Blanka Tesla, and Courtney C. Murdock. 2020. Mosquito-virus Interactions. In: J. M. Drake, M. Strand, M. Bonsall (Eds.), \textit{Population Biology of Vector-Borne Diseases.} Oxford University Press. \textit{Book Chapter}. \smallskip

\item \textcolor{awesome}{Evans, Michelle V.}, Carl W. Hintz*, Lindsey Jones*, Justine Shiau, Nicole Solano, John M. Drake, Courtney C. Murdock. 2019. Microclimate and larval habitat density predict adult \textit{Aedes albopictus} abundance in urban areas. \textit{The American Journal of Tropical Medicine and Hygiene}. doi:10.4269/ajtmh.19-0220. \smallskip

\item Kaul, RajReni B.$^\dagger$, \textcolor{awesome}{Michelle V. Evans}$^\dagger$, Courtney C. Murdock, John M. Drake. 2018. Spatio-temporal spillover risk of yellow fever in Brazil. \textit{Parasites \& Vectors} 11:488. doi: 10.1186/s13071-018-3063-6.
\smallskip

\item \textcolor{awesome}{Evans, Michelle V.}, Justine C. Shiau, Nicole Solano*, Melinda A. Brindley, John M. Drake, Courtney C. Murdock. 2018. Carry-over effects of urban larval environments on the transmission potential of dengue-2 virus. \textit{Parasites \& Vectors} 11:426. doi:10.1186/s13071-018-3013-3.

\item \textcolor{awesome}{Evans, Michelle V.}, Courtney C. Murdock, John M. Drake. 2018. Anticipating emerging mosquito-borne flaviviruses in the USA: What comes after Zika? \textit{Trends in Parasitology} 34(7):544. \linebreak doi:10.1016/j.pt.2018.02.010

\item Murdock, Courtney C., \textcolor{awesome}{Michelle V. Evans}, Taylor McClanahan*, Kerri Miazgowicz, and Blanka Tesla. 2017. Fine-scale variation in microclimate across an urban landscape changes the capacity of \textit{Aedes albopictus} to vector arboviruses. \textit{PLoS Neglected Tropical Diseases} 11(5)e0005640:, \linebreak doi:10.1371/journal.pntd.0005640.
\smallskip

\item Mordecai, Erin, Jeremy Cohen, \textcolor{awesome}{Michelle V. Evans}, Prithvi Gudapati, Leah R. Johnson, Catherine A. Lippi, Kerri Miazgowicz, et al. 2017. Detecting the impact of temperature on transmission of Zika, dengue and chikungunya using mechanistic models. \textit{PLoS Neglected Tropical Diseases} 11(4):e0005568, \linebreak doi:10.1101/063735.
\smallskip

\item \textcolor{awesome}{Evans, Michelle V.}, Tad A. Dallas, Barbara A. Han, Courtney C. Murdock, and John M. Drake. 2017. Data-driven identification of potential Zika virus vectors. \textit{eLife} 6: e22053. doi:10.7554/eLife.22053.

\end{etaremune}

\bigskip
